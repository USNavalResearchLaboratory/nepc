%TODO: use mhchem for all chemical species/equations
\documentclass[12pt]{article}
\usepackage{graphicx}
\usepackage{amsmath}
\usepackage{amssymb}
\usepackage{amscd}
\usepackage{bbm}
\usepackage{epstopdf}
\DeclareGraphicsRule{.tif}{png}{.png}{`convert #1 `dirname #1`/`basename #1 .tif`.png}
\usepackage{color}
\usepackage{rotating}
\usepackage{multirow}
\usepackage[htt]{hyphenat}
\usepackage[version=4]{mhchem}
\usepackage{subfigure}
\usepackage{dcolumn}
\usepackage[nolist,nohyperlinks]{acronym}

\newcolumntype{d}[1]{D{.}{.}{#1}}
\newcommand\mc[1]{\multicolumn{1}{c}{#1}}

%\usepackage{tikz,pgfplots}
%\pgfplotsset{compat=newest} 
%\pgfplotsset{plot coordinates/math parser=false}

% for better list... used for custom title page
\usepackage{enumitem}

%\usepackage{pgf}
%\usepackage{pgffor}
%\usepgflibrary{plothandlers}

%\usepgfplotslibrary{external} 
%\tikzexternalize% activate externalization!

% use this to remake a particular plot...
%\tikzset{external/remake next}

%use this to remake all
%\tikzset{external/force remake}

%\tikzsetexternalprefix{figures/}

%% Fonts %%
\usepackage{microtype}

\usepackage[T1]{fontenc}
\usepackage{textcomp}

% Times New Roman
\usepackage{mathptmx}

% this stuff is to make the section headings look like phys plasmas
\usepackage[scaled=.92]{helvet}
%\usepackage{helvet}

% NB, use citenum to get inline citation (not superscript)

\usepackage[colorlinks=true,
		citecolor=blue,
		linkcolor=blue,
		anchorcolor=blue,
		filecolor=blue,
		menucolor=blue,
		runcolor=blue,
		urlcolor=blue,
		unicode
		]{hyperref}
\usepackage[all]{hypcap}


%
\usepackage{geometry}
\geometry{margin=1in}

\usepackage{sectsty}  
%\sectionfont{\normalfont\sffamily\large\underline\bfseries\color{orange}}
%\sectionfont{\normalfont\sffamily\large\bfseries\color{orange}\sectionrule{0pt}{??0pt}{-4pt}{1pt}}
%\sectionfont{\normalfont\sffamily\large\bfseries\sectionrule{3ex}{3pt}{-1ex}{1pt}}
\sectionfont{\normalfont\sffamily\large\bfseries}
\subsectionfont{\normalfont\sffamily\normalsize\bfseries}

\newcommand{\mycomment}[1]{}

% set noindents for entire document
\setlength\parindent{0pt}


%%%%%%%%%%%%%%%%%%%%%%%%%%%%%%%%%%%%%%%%%%%%
%% BEGIN DOCUMENT
%%%%%%%%%%%%%%%%%%%%%%%%%%%%%%%%%%%%%%%%%%%%
\begin{document}

/Users/adamson/projects/latex/acronyms.tex

\graphicspath{{./fig/}}
\begin{titlepage}

\rmfamily
\begin{center}\sffamily\bfseries
PULSED POWER PHYSICS TECHNOTE NO. 2019-xx\\
{~}
\end{center}

\begin{description}[leftmargin=8em,style=nextline,font=\sffamily\bfseries ]
\item[TITLE:]{\bfseries 
		Notes on running UKRmol+ on \ac{nrl} workstations, \ac{nrl} \acp{hpc}, and \ac{afrl} \acp{hpc}*
}
\item[AUTHORS:]{ P.~E.~Adamson\\
{\itshape Code 6770, Plasma Physics Division, Naval Research Laboratory}}
\item[DATE:]\today
\item[ABSTRACT:] 
		The B-spline R-matrix code UKRmol+\cite{Carr2012} is used to compute electron scattering cross-sections 
		to complement data available in the literature.
\end{description}

\vfill

{\small
\noindent THIS REPORT REPRESENTS UNPUBLISHED INTERNAL WORKING DOCUMENTS AND SHOULD NOT BE REFERENCED OR DISTRIBUTED WITHOUT THE AUTHORS' CONSENT

{~}

\noindent * Work supported by DTRA/RD-NTE 6.2 program.
}
\end{titlepage}

\pagestyle{myheadings}
\markright{\hfill \color{red}\sf DRAFT VERSION DO NOT CIRCULATE \hfill}

\section{Introduction}

%TODO Intro to R-matrix calcs

The \ac{ukrmol}\cite{Carr2012} is well-established in the domain of electron—molecule 
scattering using the R-matrix method. 
In the region of the molecular target, \ac{ukrmol} uses \acp{gto} to represent 
the target and the continuum wavefunction. 
(See ref~\cite{TENNYSON201029} for a comprehensive review article on electron–molecule 
collision calculations using the \rmat{} method.) 

%TODO Intro to quantum chemistry calcs and their relationship to UKRmol

In order to compute accurate electron-atom and electron-molecule scattering cross sections, 
one must include many excited states in the quantum mechanical description (i.e. wavefunction) 
of the target (atom or molecule), and the underlying quantum chemistry calculations used to 
compute the target wavefunctions must employ large basis sets with very diffuse orbitals. 
The inclusion of the diffuse basis functions ensures an accurate description of the 
electronic spectrum of the molecule and of the scattering observables.\cite{Darby_Lewis_2017}

The inclusion of diffuse electronic states was challenging, if not impossible, in traditional 
\rmat{} scattering calculations due to numerical problems that arose when a large \ac{gto}-only 
continuum basis was combined with a large \rmat{} sphere containing the target wavefunction.
The \ac{ukrmol+} bypasses this shortfall by representing the continuum with \ac{bto} basis functions,
maintaining numerical stability with much larger \rmat{} spheres than \ac{ukrmol} and other \rmat
methods.\footnote{Zatsarinny's \ac{bsr} method achieves similar results for atomic systems. See 
ref~\cite{PhysRevA.89.062714} for an excellent example of state-of-the-art calculations of 
elastic scattering, excitation, and ionization cross sections for all transitions between 
the lowest 21 states of nitrogen in the electron energy range from threshold to 120~eV.}

\subsection{A subsection}
Lorem ipsum

\section{Acknowledgments}
blah blah blah


\bibliography{ref}{}
\bibliographystyle{unsrt}

\end{document}

\endinput


