\newcommand{\notenumber}{2019-xx}
\newcommand{\notetitle}{Background for \acp{pcm} for intense electron beam driven plasmas*}
\newcommand{\noteauthors}{P.~E.~Adamson}
\newcommand{\noteabstract}{Various \acp{pcm} are developed for intense electron
beam driven plasmas in Ar and air (dry and wet).  This work is part of an effort to 
develop \acp{prm} for a DTRA- and NRL-funded effort to update ICEPIC and MEEC++ to
model \ac{sgemp}.}
%TODO: if new calcs, describe...
%TODO: if new data from literature, describe...
%TODO: If different decisions, describe...
%TODO: How will model be V&V'd?
\newcommand{\notesponsor}{DTRA/RD-NTE 6.2 program}

%TODO: use mhchem for all chemical species/equations
\documentclass[12pt]{article}
\usepackage{graphicx}
\usepackage{amsmath}
\usepackage{amssymb}
\usepackage{amscd}
\usepackage{siunitx}
\usepackage{bbm}
\usepackage{epstopdf}
\DeclareGraphicsRule{.tif}{png}{.png}{`convert #1 `dirname #1`/`basename #1 .tif`.png}
\usepackage{color,soul}
\usepackage{rotating}
\usepackage{multirow}
\usepackage[htt]{hyphenat}
\usepackage[version=4]{mhchem}
\usepackage{subfigure}
\usepackage{dcolumn}
\usepackage{wrapfig}
\usepackage[nolist,nohyperlinks]{acronym}
\usepackage{array}
%\usepackage{parskip}
%\parskip=1em
\usepackage{verbatimbox}
\usepackage{listings}

\newcommand*{\restrictlinewidthbox}[1]{%
  \begingroup
    \sbox0{#1}%
    \ifdim\wd0>\linewidth
      \resizebox{\linewidth}{!}{\copy0}%
    \else
      \copy0 %
    \fi
  \endgroup
}

\newcolumntype{d}[1]{D{.}{.}{#1}}
\newcommand\mc[1]{\multicolumn{1}{c}{#1}}
\newcommand{\vect}[1]{\boldsymbol{#1}}

%\usepackage{tikz,pgfplots}
%\pgfplotsset{compat=newest} 
%\pgfplotsset{plot coordinates/math parser=false}

% for better list... used for custom title page
\usepackage{enumitem}

%\usepackage{pgf}
%\usepackage{pgffor}
%\usepgflibrary{plothandlers}

%\usepgfplotslibrary{external} 
%\tikzexternalize% activate externalization!

% use this to remake a particular plot...
%\tikzset{external/remake next}

%use this to remake all
%\tikzset{external/force remake}

%\tikzsetexternalprefix{figures/}

%% Fonts %%
\usepackage{microtype}

\usepackage[T1]{fontenc}
\usepackage{textcomp}

% Times New Roman
\usepackage{mathptmx}

% this stuff is to make the section headings look like phys plasmas
\usepackage[scaled=.92]{helvet}
%\usepackage{helvet}

% NB, use citenum to get inline citation (not superscript)

\usepackage[colorlinks=true,
		citecolor=blue,
		linkcolor=blue,
		anchorcolor=blue,
		filecolor=blue,
		menucolor=blue,
		runcolor=blue,
		urlcolor=blue,
		unicode
		]{hyperref}
\usepackage[all]{hypcap}

%
\usepackage{geometry}
\geometry{margin=1in}

\usepackage{sectsty}  
%\sectionfont{\normalfont\sffamily\large\underline\bfseries\color{orange}}
%\sectionfont{\normalfont\sffamily\large\bfseries\color{orange}\sectionrule{0pt}{??0pt}{-4pt}{1pt}}
%\sectionfont{\normalfont\sffamily\large\bfseries\sectionrule{3ex}{3pt}{-1ex}{1pt}}
\sectionfont{\normalfont\sffamily\large\bfseries}
\subsectionfont{\normalfont\sffamily\normalsize\bfseries}

\newcommand{\mycomment}[1]{}
\newcommand{\tbd}{\textcolor{red}{\textbf{\hl{TBD}}}}

% set noindents for entire document
\setlength\parindent{0pt}
%\setlength{\parskip}{1em}

\newcolumntype{P}[1]{>{\centering\arraybackslash}p{#1}}

%%%%%%%%%%%%%%%%%%%%%%%%%%%%%%%%%%%%%%%%%%%%
%% BEGIN DOCUMENT
%%%%%%%%%%%%%%%%%%%%%%%%%%%%%%%%%%%%%%%%%%%%
\begin{document}
/Users/adamson/projects/latex/acronyms.tex
\graphicspath{{./fig/}}
\begin{titlepage}

\rmfamily
\begin{center}\sffamily\bfseries
PULSED POWER PHYSICS TECHNOTE NO. \notenumber{}\\
{~}
\end{center}

\begin{description}[leftmargin=8em,style=nextline,font=\sffamily\bfseries ]
\item[TITLE:]{\bfseries 
\notetitle{}
}
\item[AUTHORS:]{ \noteauthors{}\\
{\itshape Code 6770, Plasma Physics Division, Naval Research Laboratory}}
\item[DATE:]\today
\item[ABSTRACT:] 
\noteabstract{}
\end{description}

\vfill

{\small
\noindent THIS REPORT REPRESENTS UNPUBLISHED INTERNAL WORKING DOCUMENTS AND SHOULD NOT BE REFERENCED OR DISTRIBUTED WITHOUT THE AUTHORS' CONSENT

{~}

\noindent * Work supported by \notesponsor{}. 
}
\end{titlepage}

\pagestyle{myheadings}



\markright{\hfill \color{red}\sf DRAFT VERSION DO NOT CIRCULATE \hfill}

\section{Introduction}

Recommended cross sections are provided, and similar criteria to
ref~\cite{itikawa2006} are used:
\begin{itemize}
\item experimental data are preferred to theoretical results, but in some cases, elaborate calculations
provide fine details which cannot be experimentally obtained;
\item agreement between independent experimental measurements of the same cross section is generally
taken as an endorsement of the accuracy of the measured data, with preference for consistency of
results taken by different techniques; and
\item cases where only a single set of data is available for a given cross section are noted
\end{itemize}

\begin{figure}
\includegraphics[width=.9\textwidth]{meichsner_fig3.7.eps}
\caption{Important elementary collision processes between particles in the plasma volume 
with collision cross section $\sigma_{AB} (\epsilon)$ ($A, B$: particles with total energy 
$\epsilon$ and charge $q$; $e$: elementary charge).\cite{meichsner2013}}
\end{figure}

\begin{figure}
\includegraphics[width=0.6\textwidth]{meichsner_fig3.8.eps}
\caption{Important elementary collision processes on the surface, $A^{+/m}$ : ion/metastable;
$R$: radical; $h \cdot \nu$: photon; $e^−$ : electron; $A, B, C$: atom or molecule; $M$: surface.\cite{meichsner2013}}
\end{figure}

\begin{table}
\caption{Overview and the Classification of the Different Elementary Collision
Processes of Electrons in the Plasma Volume}
\includegraphics[width=0.6\textwidth]{meichsner_tab3.7.eps}
\end{table}

\begin{table}
\caption{Overview and the Classification of the Different Elementary Collision
Processes of Heavy Particles in the Plasma Volume}
\includegraphics[width=0.6\textwidth]{meichsner_tab3.8.eps}
\end{table}

For the designation of the electronic energy levels of atoms and diatomic
molecules the spectroscopic notation is used:\\

\begin{itemize}
\item Atom: $nl^w\; {}^{2S+l}L_j$\\
\item Diatomic molecule: $nl^w\; {}^{2S+l}\Lambda_\Omega$
\end{itemize}
 
with the main quantum number $n$, the angular momentum $l$, the number of electrons
in the shell $w$, the resulting spin $S$, the multiplicity $2S + 1$, the resulting angular
momentum $L$ ($L = 0, 1, 2,...$ corresponding to energy levels indicating the $S, P, D,. . .$ 
states), the total angular momentum $J = L+S$ which represents the  $LS$ coupling in
the case of light atoms, and in the case of diatomic molecules  $\Omega =\Lambda + \sum_{g,u}^{+,-}$ represents
the projection of the corresponding momentum vectors onto the internuclear axis in
Greek letters with the addition + or – as well as $g$ or $u$ describing the symmetry
properties of the electronic wave function. The convention for the state assignment in
molecules are $X$ for the ground state, $A, B,\ldots$ for excited states of the same multiplicity
as the ground state $X$, and $a, b,\ldots$ for excited states of different multiplicity as $X$.

Tables and figures to include in each PCM:

\begin{itemize}
\item permitted radiative transitions in neutral atoms
\item potential energy curves
\item metastable energy levels
\item electron impact ionization thresholds
\item dissociative electron attachment thresholds
\end{itemize}

\section{Data File Formats}

\subsection{Reaction Rates}


\begin{lstlisting}[numbers=left,language=HTML,basicstyle=\small]
----------------------------------------
Reaction: e+N2(X1)->e+N2(X1)                            
Type: Ground/MomXfer					
Delta(eV): 0.00
Mass(AMU): (5.4858E-4:28.0134)->(5.4858E-4:28.0134)
Charge: (-1:0)->(-1:0)
#  These reaction rates were obtained with
#  the two-term solver Bolsig.
----------------------------------------
<E>(eV)          Rate_Constant(cm^3/s)
0.0336           4.903E-09
0.0344           4.997E-09
...
 100.3           4.515E-07
 103.4           4.533E-07
----------------------------------------
Reaction: e+N2(X1)->e+N2(Rot)
Type: Rotation
Delta(eV): 0.02
Mass(AMU): (5.4858E-4:28.0134)->(5.4858E-4:28.0134)
Charge: (-1:0)->(-1:0)
----------------------------------------
<E>(eV)          Rate_Constant(cm^3/s)
0.0336           1.804E-11
0.0344           1.849E-11
...
\end{lstlisting}


\bibliography{ref}{}
\bibliographystyle{unsrt}

\end{document}

\endinput



