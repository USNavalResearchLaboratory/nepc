The main processes in nonthermal plasmas operating in oxygen, nitrogen, or dry air
plasma are dissociative collisions of molecules, resulting in the generation of the
reactive atoms (O, N) [1,2], the formation of excited atoms and molecules, as well
as positive or negative ions. The formation of negative ions is essential mainly for
electronegative gases such as oxygen. The dissociative attachment of electrons of
excited O 2 molecules generates negative atomic ions as well as oxygen atoms. The
threshold energy of this process is essentially lower than electron impact dissociation
and dissociative ionization of ground state molecules [3]. The reaction probability of
heavy particle reactions of electronically excited species can exceed the probabilities
of ground state reactions by orders of magnitude [4,5].
The air plasma chemistry, e.g., is responsible for producing N x O y compounds,
which have a key role in global environmental problems like acid rain. The scheme
in Figure 2.1 of dominant plasma chemical reactions in dry air demonstrates the
complexity of the processes [6].
The plasma chemistry in oxygen is also of practical importance, namely, for
the ozone generation and for plasma ashing. Augmented combustion is essentially
influenced by air plasma chemistry [7].

meichsner_fig2.1.png
Diagram of primary chemical reactions in dry air plasma induced by electron
impact. (According to Becker, K.H. et al., Air plasma chemistry, in Becker, K.H. et al. (eds),
Non-Equilibrium Air Plasmas at Atmospheric Pressure, IoP, Bristol, U.K., pp. 124–182, 2005.)


in the case of electronegative gases, the electron may be captured and
a stable negative ion is produced, e.g., the dissociative electron attachment reaction
for oxygen molecule.

$e^− + O_2 \rightarrow O^− + O, \varepsilon_{thres} ∼ 4 eV (T_g = 300 K)$
